\documentclass[11pt]{article}

\usepackage[margin=2cm]{geometry}
\usepackage{amsmath,amsthm,amssymb}
\usepackage{subcaption}
\usepackage{float,graphicx}
\usepackage{hyperref}
\usepackage{enumerate}
\usepackage[margins,movemargins]{trackchanges}

\hypersetup{
  colorlinks   = true, %Colours links instead of ugly boxes
  urlcolor     = blue, %Colour for external hyperlinks
  linkcolor    = blue, %Colour of internal links
  citecolor   = red %Colour of citations
}

\newsavebox{\imagebox}

\addeditor{gh}

% Bibliography
\usepackage[
backend=biber,
style=apa,
% maxbibnames=10,
maxcitenames=1,
% natbib=true,
% giveninits=true,
]{biblatex}
\addbibresource{ohbemn.bib}

\title{Oh Bemn!}
\author{Gaute Hope}
\date{2024}

\begin{document}
\maketitle

\section{Ocean waves}

Small wave approximation, $\Phi$ is the velocity potential:

\begin{eqnarray}
  u = \frac{\partial \phi}{\partial x} \\
  w = \frac{\partial \phi}{\partial y}
\end{eqnarray}

where $u$ and $w$ is the water particle velocity in $x$ and $y$ direction.

At the surface:

\begin{equation}
  \frac{\partial \phi}{\partial t} + g \eta = 0; z = 0
\end{equation}

\textcite{holthuijsen2010WavesOceanicCoastal} Eq. 5.3.29.


\subsection{Boundary conditions}

At the seabed the vertical velocity is $0$:

\begin{equation}
  \frac{\partial \phi}{\partial z} = 0; z = -d
\end{equation}


\section{Boundary Element Method}
\textcite{kirkup1998BoundaryElementMethod}

\clearpage
\printbibliography

% \appendix
% \section{}
% \label{app:app1}


\end{document}

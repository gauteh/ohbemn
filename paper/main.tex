\documentclass[11pt]{article}

\usepackage[margin=2cm]{geometry}
\usepackage{amsmath,amsthm,amssymb}
\usepackage{subcaption}
\usepackage{float,graphicx}
\usepackage{hyperref}
\usepackage{enumerate}
\usepackage[margins,movemargins]{trackchanges}

\hypersetup{
  colorlinks   = true, %Colours links instead of ugly boxes
  urlcolor     = blue, %Colour for external hyperlinks
  linkcolor    = blue, %Colour of internal links
  citecolor   = red %Colour of citations
}

\newsavebox{\imagebox}

\addeditor{gh}

% Bibliography
\usepackage[
backend=biber,
style=apa,
% maxbibnames=10,
maxcitenames=1,
% natbib=true,
% giveninits=true,
]{biblatex}
\addbibresource{ohbemn.bib}

\title{Oh Bemn!}
\author{Gaute Hope}
\date{2024}

\begin{document}
\maketitle

\section{Ocean waves}

Small wave approximation, $\phi$ is the velocity potential:

\begin{eqnarray}
  \mathbf{u} = \nabla \phi \\
  u = \frac{\partial \phi}{\partial x} \\
  w = \frac{\partial \phi}{\partial y}
\end{eqnarray}

where $u$ and $w$ is the water particle velocity in $x$ and $y$ direction.

The flow is irrotational:

\begin{equation}
  \nabla \times \phi = 0
\end{equation}

And incompressible:

\begin{equation}
  \nabla \cdot \phi = 0
\end{equation}

In deep water, assuming a flat bottom, the Laplace-equation describes ocean waves:

\begin{equation}
  \nabla^2 \phi = 0 \label{eq:wave-eq-laplace}
\end{equation}

See \textcite{ardhuin2023OceanWavesGeosciences} (Part I). In intermediate or
shallow waters with varying depth we get the Helmholtz-equation, with a
dependence on $k$:

\begin{equation}
  \nabla^2 \phi + k^2 \phi = 0 \label{eq:wave-eq-helmholtz}
\end{equation}

See \textcite{ardhuin2023OceanWavesGeosciences} (Part II).

\subsection{Boundary conditions}

\subsubsection{Kinematic boundary conditions}

At the surface, the pressure is equal to the atmospheric pressure:

\begin{equation}
  \frac{\partial \phi}{\partial t} + g \eta = 0; z = 0
\end{equation}

\textcite{holthuijsen2010WavesOceanicCoastal} Eq. 5.3.29.

Relating the surface to the velocity potential, linear approximation:

\begin{equation}
  \frac{\partial \eta}{\partial t} = \frac{\partial \phi}{\partial z} \bigg\rvert_{z=\eta}
\end{equation}

This is defined at the varying surface, using Taylor-expansion and disregarding the quadratic terms:

\begin{equation}
\frac{\partial \eta}{\partial t} = \frac{\partial \phi}{\partial z} \bigg\rvert_{z=0} \label{eq:surface-kinematic-bc}
\end{equation}

% https://falk.ucsd.edu/pdf/WavesLecture02_211A.pdf

Equation \eqref{eq:surface-kinematic-bc} is the surface kinematic boundary condition.

At the seabed the vertical velocity is $0$:

\begin{equation}
  \frac{\partial \phi}{\partial z} = 0; z = -d
\end{equation}

\subsubsection{Dynamic boundary condition}

\begin{equation}
  \frac{\partial \phi}{\partial t} = -g \eta \bigg\rvert_{z=0}
\end{equation}


% \subsubsection{Summary}

% https://falk.ucsd.edu/pdf/WavesLecture02_211A.pdf

\subsection{Boundary Conditions between interior and exterior}

\subsubsection{Water-water with different depth}

The wave speed changes according to the dispersion relation
(\eqref{eq:dispersion}). The velocity across the boundary must be
continuous:

\begin{equation}
  \frac{\partial \phi_1}{\partial n} = \frac{\partial \phi_2}{\partial n}
\end{equation}

Otherwise the amplitude ($\propto \phi$) is free to change.

\subsubsection{Water-solid}

E.g. a break-water. The particle velocity across the boundary (at the boundary) is zero:

\begin{equation}
\frac{\partial \phi}{\partial n} = 0 \bigg\rvert_{x \in S}
\end{equation}

\section{Solution to the linear wave equation}

A plane wave:

\begin{equation}
  \phi = A \exp{(i\mathbf{k \cdot x})}
\end{equation}

Solution to plane wave traveling along $x$:

\begin{equation}
  \eta(x,y,t) = a \sin{(\omega t - kx)}
\end{equation}

\begin{equation}
  \hat\phi(x, y, z, t) = \frac{ag}{\omega} \frac{\cosh{(k(z+h))}}{\cosh{(kh)}}
\end{equation}

\subsection{Dispersion relation}

\begin{equation}
  w^2 = gk \tanh{(kh)} \label{eq:dispersion}
\end{equation}

\section{Boundary Element Method}

The Laplace equation, \eqref{eq:wave-eq-laplace}, is transformed to the integral equation by applying Green's second theorem:

\begin{equation}
  \int_S \frac{\partial G}{\partial n_q}(\mathbf{p}, \mathbf{q}) \phi(\mathbf{q}) d S_q + \frac{1}{2} \phi(q) = \int_S G(\mathbf{p}, \mathbf{q}) \frac{\partial \phi}{\partial n_q} dS_q \label{eq:integral-equation}
\end{equation}

Typically we are given on the boundary the values for either $\phi(\mathbf{x})$
(Dirichlet) or $\frac{\partial \phi}{\partial n_q}$ (Neumann) on the boundary \parencite{kirkup1998BoundaryElementMethod}.

\subsection{Operator notation}

$\xi$ is a function on the boundary $S$, with $\mathbf{p}$ being all points on $S$:

\begin{equation}
  \int_S G(\mathbf{p}, \mathbf{q}) \xi(q) dS_q = v(\mathbf{p})
\end{equation}

then the $L$ operator is:

\begin{equation}
  \{L \xi \}_S (\mathbf{p}) = v(\mathbf{p})
\end{equation}

% TODO: add M operator.

The integral equation, \eqref{eq:integral-equation}, linearised and in operator notation is:

\begin{equation}
  M \hat\phi + \frac{1}{2} \hat\phi = L \hat v \label{eq:integral-operator-form}
\end{equation}

Where $\hat \phi$ and $\hat v$ are approximations to $\phi(\mathbf{x})$
and $\frac{\partial \phi}{\partial n_q}$ respectively at a set of points on the boundary $S$.

\subsection{Boundary condition across elements}

Across the boundary the velocity must be continuous:

\begin{equation}
  \mathbf{u_0}(\Omega) \cdot \mathbf{n} = \mathbf{u_1}(\Omega) \cdot \mathbf{n}
\end{equation}

\begin{equation}
  \frac{\partial \phi}{\partial \mathbf{n}} = 0
\end{equation}

Which is a \emph{Neumann boundary condition}: specifying the value of the \emph{derivative} of $\phi$ on the boundary.

\subsection{Domain solution}

\begin{equation}
  \phi(\mathbf{x}) = \{L \phi\}_S - \{M v\}_S \quad \mathbf{x} \in D
\end{equation}

\subsection{Colocation}

The colocation method approximates the boundary function by a constant at each segment:

\begin{eqnarray}
  \phi(\mathbf{p}) \approx \phi_j \\
  v(\mathbf{p}) \approx v_j
\end{eqnarray}

\section{Source}

Page 32 in \textcite{kirkup1998BoundaryElementMethod}.

\subsection{Boundary derivatives}

Page 34 in \textcite{kirkup1998BoundaryElementMethod}.


The directional derivative of a function $f$ in the direction of the unit
vector $\mathbf{n}$ is found through the \emph{gradient} and the directional
vector:

\begin{equation}
  \nabla_\mathbf{n} f = \nabla f(\mathbf{x}) \cdot \mathbf{n}
\end{equation}

A plane wave:

\begin{equation}
  \phi = A \exp{i (k_x x + k_y y)}
\end{equation}

The gradient is:

\begin{equation}
\end{equation}

\begin{equation}
  \nabla \phi
  = A \begin{bmatrix}
    -ik_x \\
    -ik_y \\
  \end{bmatrix}
  \cdot \exp{i(k_x x + k_y y)}
\end{equation}

\clearpage
\printbibliography

% \appendix
% \section{}
% \label{app:app1}



\section{Cases}

\subsection{Rectangular interior modes}
\begin{figure}[ht]
  \centering
  \includegraphics[width=0.8\textwidth]{../examples/figures/rectangular_interior.png}
  \caption{Rectangular interior}
  \label{fig:rectangular-interior}
\end{figure}
\end{document}
